\chapter{Implementacija i korisničko sučelje}
		
		
		\section{Korištene tehnologije i alati}
		
			Komunikacija unutar tima realizirana je korištenjem aplikacija \underline{WhatsApp} i \underline{Discord}. Za izradu UML dijagrama korišten je alat \underline{Astah Professional}. Kao sustav za upravljanje izvornim kodom upotrebljavali smo \underline{Git}, a udaljeni repozitorij projekta je dostupan na web platformi \underline{GitHub}.

			Kao razvojna okruženja korišteni su \underline{Android Studio} i \underline{PyCharm}. Android Studio je integrirano razvojno okruženje za Googleov operativni sustav Android, izgrađeno na JetBrainsovom IntelliJ IDEA softveru i dizajnirano posebno za Android razvoj. Dostupan je za preuzimanje na Windows, macOS i Linux operativnim sustavima. PyCharm  je integrirano razvojno okruženje koje se koristi za programiranje u Pythonu koji je razvila tvrtka JetBrains. Omogućuje analizu koda, integrirani tester jedinica \textit{(engl. unit testing)}, grafički \textit{debugger} i podržava web razvoj s Djangom. Isto kao i Android Studio, dostupan je na različitim operacijskim sustavima.

			Aplikacija je napisana koristeći radni okvir \underline{FastAPI} i jezik \underline{Python} za izradu \textit{backenda} te jezik \underline{Kotlin} za izradu \textit{frontenda}. FastAPI je moderan web okvir za izgradnju RESTful API-ja u Pythonu. Popularan je među programerima zbog svoje jednostavnosti, robusnosti i brzine.

			Baza podataka se nalazi na poslužitelju u \underline{Renderu}.
			
			\eject 
		
	
		\section{Ispitivanje programskog rješenja}
			
			\textbf{\textit{dio 2. revizije}}\\
			
			 ...
	
			
			\subsection{Ispitivanje komponenti}
			...
			
			
			
			\subsection{Ispitivanje sustava}
			
			...
			 
			
			\eject 
		
		
		\section{Dijagram razmještaja}
			
			\textbf{\textit{dio 2. revizije}}
			
			 ...
			
			\eject 
		
		\section{Upute za puštanje u pogon}
		
			\textbf{\textit{dio 2. revizije}}\\
		
			...
			
			
			\eject 