\chapter{Zaključak i budući rad}
		
		
		Naš zadatak bio je razviti aplikaciju za pronalaženje nestalih ljubimaca s mogućnošću komunikacije oko potrage, pregledom aktivnih i neaktivnih oglasa te mogućnošću uređivanja oglasa i funkcionalnost korisnika registriranih kao sklonište. Tijekom 15 tjedana rada u timu, ostvarili smo postavljeni cilj kroz dvije ključne faze.
		
		U prvoj fazi projekta, okupili smo tim, odabrali projektni zadatak i posvetili se intenzivnom dokumentiranju zahtjeva. Također smo i jednoglasno odabrali vođu tima koji je obavljao razgovor s asistentom i profesorom predmeta. Odlučili smo se za razvoj android aplikacije jer nam se činila kao najbolje rješenje za zadani problem. Tim smo podijelili u backend, frontend i tim za dokumentaciju. Pisanje dokumentacije i izrada raznih dijagrama pružili su smjernice podtimovima za razvoj backenda i frontenda. Naravno da smo imali nesuglasica, ali njih bi brzo riješili sastancima uživo koje smo imali često.
		
		U drugoj fazi je svaki član tima znao na čemu treba raditi, pa je bilo više samostalnog rada nego u prvoj fazi. Uz pomoć ostalih članova tima i raznih materijala smo zajedno prelazili prepreke na koje bi naišli. Komunikacija je bila ključna jer smo u ovoj fazi mnogo ovisili o ostalim članovima tima i njihovom radu. Na kraju druge faze smo radili razne testove kako bi osigurali da aplikacija radi kako je i zamišljena.
		
		Komunikacija među članovima odvijala se putem Whatsappa i Discorda kojeg je frontend najviše koristio za videopozive. Sudjelovanje u projektu bilo je vrijedno iskustvo za članove tima, naglašavajući važnost dobre organizacije i suradnje. Unutar tima smo imali studente s različitim prijašnjim iskustvima na sličnim projektima te su oni s više znanja uvijek bili voljni pomoći. Unatoč prostoru za usavršavanje, postignuto je značajno napredovanje u razumijevanju i implementaciji aplikacije.
		
		\eject 